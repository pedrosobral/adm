\chapter{Limites}
\thispagestyle{empty}

\section{Introdução}
Seja $f\left(x\right)$ uma função com variável real. Em alguns contextos
é importante compreender o comportamento da função $f$ quando a variável
$x$ se aproxima de um determinado valor $a$ (diremos ``$x$ tende
a $a$''). Quando, a medida que $x$ tende a $a$, o valor de $f\left(x\right)$
está se aproximando de um número $L$, diremos que ``$L$ é o limite
de $f\left(x\right)$ quando $x$ tende a $a$'' e denotaremos

\[
\limitex{a}{f(x)}=L.
\]


Um exemplo típico de cálculo de limite é a função $f\left(x\right)=\frac{x^{2}-1}{x-1}$.
Note que neste caso o denominador $x-1$ se anula quando $x=1$, portanto
$f\left(x\right)$ não está definida para $x=1$.

Neste caso a informação de que a medida que $x$ se aproxima de $1$,
$f\left(x\right)$ se aproxima de $2$ é de grande importância, como
veremos nos próximos capítulos.


\subsection{Definição matemática de limite}

Aqui, queremos tornar preciso o conceito de aproximação implícito
na sentença

\begin{center}
\emph{Quando $x$ tende a $a$ $f\left(x\right)$ tende a $L$, }
\par\end{center}

\begin{flushleft}
que aparece quando falamos de limite.
\par\end{flushleft}

Dizer que $x$ se aproxima de $a$ significa que a distancia entre
$x$ e $a$ fica pequena. Isto é,
\[
\left|f\left(x\right)-L\right|
\]
 fica cada vez menor.

Tendo estes conceitos em mente, a definição precisa de limite é

\begin{defn}
Diremos que $L$ é o limite de $f\left(x\right)$ quando $x$ tende
a $a$ se dado $\epsilon>0$ existe $\delta>0$ tal que
\[
\left|x-a\right|<\delta
\]
implica que $\left|f\left(x\right)-L\right|<\epsilon$. Escreveremos
\[
L=\limitex{a}{f(x)}
\]
\end{defn}



\begin{exemplo}
Verifique que $$\limitex{1}{3x-1}=2.$$

Devemos usar a definição anterior. Ou seja, dado um $\epsilon>0$
qualquer, devemos ser capazes de encontrar um outro número $\epsilon>0$
tal que
\begin{equation}
\left|x-1\right|<\epsilon\mbox{ implica que}\left|\left(3x-1\right)-2\right|<\epsilon\tag{1}
\end{equation}


Note que
\[
\left|3x-1-2\right|=\left|3x-3\right|=3\left|x-1\right|.
\]


Portanto,
\[
\left|x-1\right|<\frac{\epsilon}{3}\Rightarrow3\left|x-1\right|<\epsilon\Rightarrow\left|\left(3x-1\right)-2\right|<\epsilon.
\]


Ou seja, podemos tomar $\delta=\frac{\epsilon}{3}$ na sentença (1).
Fica assim demonstrado que
\[
\limitex{1}{\left(3x-1\right)}=2.
\]
\end{exemplo}
